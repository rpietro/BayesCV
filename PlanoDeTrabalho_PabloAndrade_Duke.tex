\documentclass{article}

\usepackage[process=auto]{pstool}
\usepackage[brazil]{babel}
\usepackage[latin1]{inputenc}
\usepackage[T1]{fontenc}
\usepackage{slashbox}
\usepackage{subfigure}
\usepackage{multirow}
\usepackage{fancybox}
\usepackage{caption}
\usepackage{color}
\usepackage[fleqn]{amsmath}
\title{Plano de trabalho para Doutorado Sandu?che no Exterior - Duke University/EUA}
\usepackage{hyperref}

\begin{document}

\author{Pablo de Morais Andrade\\\\
		Universidade de São Paulo-USP,\\
  		São Paulo/SP,\\
  		Brasil,\\\\
  \texttt{pablo.andrade@usp.br}}
\maketitle

\begin{abstract}
O presente plano de trabalho tem como objetivo a cadidatura do estudante de doutorado em Bioinformática da 
Universidade de São Paulo-USP, Pablo de Morais Andrade a uma bolsa de doutorado sanduíche no exterior 
a partir do programa Ciência sem Fronteiras. 
Esse documento descreve o objetivo do programa assim como as atividades a serem desenvolvidas pelo 
estudante no exterior e os resultados esperados.
\end{abstract}

\section{ \label{sec:projeto} Candidato e Orientador}
O candidato Pablo de Morais Andrade é Bacharel em Engenharia da Computação pela Universidade Estadual de 
Campinas-Unicamp, Mestre em Ciência da Computação pela Universidade do Missouri-Kansas City e atualmente 
aluno no segundo ano de doutorado em Bioinformática na Universidade de São Paulo-USP.\\
\\
O orientador Carlos Alberto de Bragança Pereira é Professor Titular de Estatística da Universidade de São Paulo-USP. 
Recebeu Diversas bolsas de estudos e auxílios de pesquisas da FAPESP, do CNPq e da CAPES. 
Recebeu "Grants" da University of Pittsburgh, University of California at Berkeley, 
International Society of Environmetrics, Organization of American States. 
Ganhou o Prêmio "Ralph Bradley Award" (Ph.D. Thesis) de melhor tese de PhD na Florida State University. 
Recebeu o prêmio "Appreciation Award: Oak Ridge Associated Universities". Membro Eleito do ISI, 
the International Statistical Institute. Fellow do Institute of Statisticians. 
Membro do Comitê de pesquisas do Instituto Superior Técnico de Lisboa. Membro do Comitê Cietífico da 
Environmetrics 2001, Oregon.\\
Estre as publicações do orientador Carlos Alberto de Bragança Pereira estão 3 livros (Português) e 
de mais de 100 artigos de pesquisa em diversos periódicos internacionais incluindo: 
The American Statistician, Biometrika, Applied Statistics, Journal of Statistical Planning and Inference, 
The Statistician, Network, Environmetrics, Test, Sankhyã, Communications in Statistics, 
International Statistical Review, Journal of Statistical Computation and Simulation, Mutation Research, 
Journal of General Virology; Skandinavian Journal of Statistics, Acta Cytologica, Biometrical Journal, 
Journal of Multivariate Analysis, Teratogenesis, Carcinogenesis and Mutagenesis, Eur. Journal Vasc. Endovasc. Surg., 
International Journal of Mathematical and Statistical Science, Medicine, Environmental Health Perspectives, 
Genetics e Entropy.

\section{ \label{sec:duke} Grupo de trabalho na Universidade Duke}
O departamento de cirurgia o da Universidade de Duke é um dos líderes em cirurgia no mundo. 
Além de contar com médicos e professores de excelência global, possui grande experiência e consequentemente 
informações disponíveis tanto a partir de coleta de dados como opniões de especialistas na área de cirurgia.
\section{ \label{sec:projeto} Projeto}

\subsection{ \label{subsec:motivacao} Motivação}
Doenças cardiovasculares representam um grande problema de saúde pública e sendo os recursos limitados, 
se faz necessário uma orientação na tomada de decisões para os médicos a fim de fazer o melhor uso possível dos 
recursos disponíveis.\\ 
A criação de um modelo para tomada de decisão relacionado a doenças cardiovasculares poderia viabilizar esse 
direcionamento para os médicos além de possibilitar a disseminação do uso de tecnologia nas políticas de saúde e
aprimorar a qualidade e eficiência da medicina carviovascular no Brasil.\\
Pesquisas desenvolvidas para criação de modelos para tomadas de decisão relacionada à medicina são, na sua maioria, 
conduzidas dentro de um contexto cultural e econômico local. Por isso, se faz necessário um estudo a partir da 
análise de dados disponíveis pelas instituições Brasileiras ligados à cardiologia.\\
O departamento de cirurgia da Universidade de Duke nos Estados Unidos está começando um trabalho juntamente com 
a Sociedade Brasileira de Cardiologia na criação de tais modelos, e o doutorado sanduíche teria como principal foco, 
o trabalho conjunto na pesquisa e desenvolvimento desses modelos.\\
Além disso, existe no momento uma oportunidade disponível para análise de de dados com o objetivo
de estimar a mudança na probabilidade de mortalidade no período perioperatório, quando levados em consideração os 
fatores de riscos perioperários, interoperatórios e pósoperatórios, para cirurgias cardíacas.\\
O grupo de cirurgia da Universidade de Duke tem um grande número de bancos de dados, que estariam disponíveis 
para análise pelo aluno quando esse estivesse trabalhando localmente (na Universidade de Duke).

\subsection{ \label{subsec:cardiovascular} Tomadas de Decisão para condições cardiovasculares no Brasil}
O projeto a ser desenvolvido pelo estudante na Universidade de Duke tem como objetivo analisar e direcionar tomadas de decisões relacionadas a problemas cardiovasculares - como insuficiência cardíaca crônica -  no Brasil. Esse trabalho será realizado conjuntamente com o departamento de cirurgia da Universidade de Duke, e será baseado nos bancos de dados relacionados a cardiologia disponíveis.\\
Um modelo para tomada de decisão deve combinar a literatura disponível, bancos de dados que representam as condições atuais cardiovasculares em diferentes regiões do Brasil e a opnião de especialistas na área, além de possibilitar a atualização do modelo a medida que mais informação seja disponibilizada.\\
A fim de atingir esses objetivos, as seguintes tarefas devem ser executadas:
\begin{itemize}
	\item Estudo dos dados relacionados à cardiologia atualmente existentes no Brasil e modelos que utilizam esses dados.
	\item Criação de um banco de dados com toda a informação relacionada à cardiologia disponível no Brasil.
	\item Estudo detalhado dos modelos atualmente utilizados em outros países para tomada de decisão relacionadas a cardiologia.
	\item Buscar o melhor modelo a ser utilizado especificamente para os dados de cardiologia no Brasil.
	\item Conduzir uma sequência de testes usando os dados disponíveis para validar o modelo escolhido, se necessário, rever o modelo e refazer os testes.
	\item Discutir com especialistas na área de cardiologia a validade do método e seus resultados.
\end{itemize}
Uma vez decidido o modelo a ser utilizado, esse deve permitir a entrada de informações fornecidas por especialistas e 
atualização do modelo a partir do dados. Dessa forma, um modelo Bayesiano é considerado, já que as duas etapas podem 
ser utilizadas no modelo da seguinte forma:
 \begin{enumerate} 
	\item Entrada da informação \textit{a priori}: o modelos é parametrizado de acordo com a informações dadas 
		pelos especialistas.
	\item \textit{Calibração} do modelo: a partir da entradas dos dados a partir dos bancos de dados existentes, 
o modelo é calibrado (os parâmetros são ajustados para que o modelo se adeque aos dados).
 \end{enumerate}
A criação de tal modelo tem como objetivo final garantir que a melhor prática na tomada de decisão seja utilizada em 
hospitais brasileiros e que esse modelo contenha as característica locais do país. Para isso, contaremos com a 
disseminação por parte de parceiros envolvidos no projeto, que estão diretamente envolvidos com cardiologia no Brasil.\\
O modelo resultante e sua disseminação deve ser trazido para 
uma ampla discução envolvendo membros de hospitais de diversas área do Brasil. 
Esse membros devem proceder à criticar e propor mudanças ao modelo até que um modelo final que satisfaça as 
necessidades de seus grupos seja encontrado e esse possa ser passado para os membros de seus respectivos grupos. 
O modelo deve apresentado na conferência anual da Sociedade Brasileira de Cardiologia.\\
Apesar do projeto ser primeiramente focado apenas em problemas de cardiologia, se espera que tenhamos ao final do 
projeto uma metodologia que possa ser estendida a outras disciplinas e outros modelos possam ser criados para auxiliar 
médicos de outras área em hospitais brasileiros.
\subsection{ \label{subsec:modelos} Modelos}
Os modelos a serem estudados e aplicados aos dados devem incluir:
\begin{enumerate}
\item Árvores de decisão 
\item Redes Bayesianas 
\item Campos aleatórios de Markov
\item Modelo oculto de Markov
\end{enumerate}
Esse modelos têm como vantagem, o fato de apresentarem as duas características desejadas: possibilidade de entradas de 
informações dos especialitas \textit{a priori} e possibilidade de atualização do modelo a partir de \textit{aprendizado}
utilizando dos dados disponíveis e novos dados que podem ser entrados no sistema.\\
Usando esse modelo é possível fazer avaliação da qualidade dos serviços prestados, análise custo-benefício na gestão 
de recursos públicos além de pesquisa de aderência às diretrizes da Sociedade Brasileira de Cardiologia em hospitais 
da rede pública e privada.\\
O modelo de Redes Bayesianas, especificamente, que é o foco de estudos do aluno no doutorado, têm sido vastamente 
explorado na análise de dados biológicos (\cite{friedman_2000}). Os modelos preditivos usando redes Bayesianas levam 
em consideração não somente a correlação entre cada variável (fator) e a variável objetivo (a ser estimada), 
mas também a correlação entre as variáveis, evitando assim uma super-penalização ou super-estimação de variáveis 
altamente correlacionadas(\cite{friedman_1997}).\\
No caso específico de cirurgias cardíacas, as variáveis em questão são fator de risco periopratórios, interoperatórios 
e posoperatórios e como esses fatores influenciam na probabilidade de morte do paciente durante a cirurgia cardíaca.

\subsection{ \label{sec:resultados} Resultados Esperados}
Ao final do programa devem-se ter modelos de análises de decisão relacionados a problemas cardiovasculares, 
para auxiliar médicos em hospitais brasileiros. Além disso deve ser criada uma metodologia de pesquisa, 
desde as etapas iniciais até a aplicação desses métodos, para possível reprodução da pesquisa e projeto em 
outras áreas da medicina brasileira.

\subsection{ \label{sec:cronograma} Cronograma}
O cronograma do doutorado saduíche inclue 12 meses na Universidade de Duke, divididos da seguinte forma:\\
{\color{red} ... acrescentar info ...}

%isso daqui nos vamos decidir durante a reuniao. uma vez que voce tenha um manuscrito mais completo dai eu reviso com voce de novo

\clearpage

\begin{thebibliography}{9}
\bibitem{bocchi_2012}
Bocchi EA, Marcondes-Braga FG, Bacal F, Ferraz AS, Albuquerque D, Rodrigues D, et al. Sociedade Brasileira de Cardiologia. Atualização da Diretriz Brasileira de Insuficiência Cardíaca Crônica - 2012. Arq Bras Cardiol 2012: 98(1 supl. 1): 1-33
\bibitem{montera_2009}
Montera MW, Almeida RA, Tinoco EM, Rocha RM, Moura LZ, Réa-Neto A, et al.
Sociedade Brasileira de Cardiologia. II Diretriz Brasileira de Insuficiência Cardíaca Aguda.
Arq Bras Cardiol.2009;93(3 supl.3):1-65
\bibitem{friedman_1999}
Friedman, N., Goldszmidt, M., and Wyner, A. 1999. Data analysis with Bayesian networks: A bootstrap approach.
Proc. Fifteenth Conference on Uncertainty in Artificial Intelligence (UAI ’99), 206–215.
Friedman, N., and Koller, D. 2000. Being Bayesian about network structure. Proc. Sixteenth Conference on Uncertainty in Artificial Intelligence (UAI ’00), 201–210.
\bibitem{friedman_1997}
Friedman, N., Geiger, D., \& Goldszmidt, M. (1997). Bayesian network classifiers. Machine Learning, 29, 131-163.
\bibitem{friedman_2000}
Friedman, N., Linial, M., Nachman, I., \& Peer, D. (2000). Using Bayesian networks to analyse expression data. Computational Biology, 7, 601-620.
\end{thebibliography}

\end{document}