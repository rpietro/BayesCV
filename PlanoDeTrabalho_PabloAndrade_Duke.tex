\documentclass{article}

\usepackage[process=auto]{pstool}
\usepackage[brazil]{babel}
\usepackage[latin1]{inputenc}
\usepackage[T1]{fontenc}
\usepackage{slashbox}
\usepackage{subfigure}
\usepackage{multirow}
\usepackage{fancybox}
\usepackage{caption}
\usepackage{color}
\usepackage[fleqn]{amsmath}
\title{Plano de trabalho para Doutorado Sandu?che no Exterior - Duke University/EUA}
\usepackage{hyperref}

\begin{document}

\author{Pablo de Morais Andrade\\\\
		Universidade de S?o Paulo-USP,\\
  		S?o Paulo/SP,\\
  		Brasil,\\\\
  \texttt{pablo.andrade@usp.br}}
\maketitle

\begin{abstract}
O presente plano de trabalho tem como objetivo a cadidatura do estudante de doutorado em Bioinform?tica da Universidade de S?o Paulo-USP, Pablo de Morais Andrade a uma bolsa de doutorado sandu?che no exterior a partir do programa Ci?ncia sem Fronteiras. Esse documento descreve o objetivo do programa assim como as atividades a serem desenvolvidas pelo estudante no exterior e os resultados esperados.
\end{abstract}

\section{ \label{sec:projeto} Candidato e Orientador}
O candidato Pablo de Morais Andrade ? Bacharel em Engenharia da Computa??o pela Universidade Estadual de Campinas-Unicamp, Mestre em Ci?ncia da Computa??o pela Universidade do Missouri-Kansas City e atualmente aluno no segundo ano de doutorado em Bioinform?tica na Universidade de S?o Paulo-USP.\\

%legal que voce ja morou aqui, vai facilitar muito a comunicacao
\\
O orientador Carlos Alberto de Bragan?a Pereira ? Professor titular da Universidade de S?o Paulo-USP, {\color{red} ... acrescentar info ...}

\section{ \label{sec:duke} Grupo de trabalho na Universidade Duke}
O grupo \textit{Research on Research} da Universidade de Duke {\color{red} ... acrescentar info ...}

%a Dida pode te ajudar com referencias ao Department of Anesthesiology

\section{ \label{sec:projeto} Projeto}

\subsection{ \label{subsec:motivacao} Motiva??o}
Doen?as cardiovasculares representam um grande problema de sa?de p?blica e sendo os recursos limitados, se faz necess?rio uma orienta??o na tomada de decis?es para os m?dicos a fim de fazer o melhor uso poss?vel do recursos dispon?veis.\\ 
A cria??o de um modelo para tomada de decis?o relacionado a doen?as cardiovasculares poderia viabilizar esse direcionamento para os m?dicos al?m de possibilitar a dissemina??o do uso de tecnologia nas pol?ticas de sa?de e aprimorar a qualidade e efici?ncia da medicina carviovascular no Brasil.\\
Pesquisas desenvolvidas para cria??o de modelos para tomadas de decis?o relacionada ? medicina s?o, na sua maioria, conduzidas dentro de um contexto cultural e econ?mico local. Por isso, se faz necess?rio um estudo a partir da an?lise de dados dispon?veis pelas institui??es Brasileiras ligados ? cardiologia.\\
O grupo "Research on Research" da Universidade de Duke nos Estados Unidos est? come?ando um trabalho juntamente com a Sociedade Brasileira de Cardiologia na cria??o de tais modelos, e o doutorado sandu?che teria como principal foco, o trabalho conjunto na pesquisa e desenvolvimento desses modelos.
\subsection{ \label{subsec:evidencemed} Medicina baseada em evid?ncia}
Medicina baseada em evid?ncia {\color{red} ... acrescentar info ...}

\subsection{ \label{subsec:cardiovascular} Tomadas de Decis?o para condi??es cardiovasculares no Brasil}
O projeto a ser desenvolvido pelo estudante na Universidade de Duke tem como objetivo analisar e direcionar tomadas de decis?es relacionadas a problemas cardiovasculares - como insufici?ncia card?aca cr?nica -  no Brasil. Esse trabalho ser? realizado conjuntamente com o grupo "Research on Research" da Universidade de Duke, e ser? baseado nos bancos de dados relacionados a cardiologia dispon?veis.\\
Um modelo para tomada de decis?o deve combinar a literatura dispon?vel, bancos de dados que representam as condi??es atuais cardiovasculares em diferentes regi?es do Brasil e a opni?o de especialistas na ?rea, al?m de possibilitar a atualiza??o do modelo a medida que mais informa??o seja disponibilizada.\\
A fim de atingir esses objetivos, as seguintes tarefas devem ser executadas:
\begin{itemize}
	\item Estudo dos dados relacionados ? cardiologia atualmente existentes no Brasil e modelos que utilizam esses dados.
	\item Cria??o de um banco de dados com toda a informa??o relacionada ? cardiologia dispon?vel no Brasil.
	
	%por favor mande um email para a "Clarissa Garcia Rodrigues" <cgr7@duke.edu>,  e ela vai te mandar informacoes a respeito de bancos brasileiros
	
	\item Estudo detalhado dos modelos atualmente utilizados em outros pa?ses para tomada de decis?o relacionadas a cardiologia.
	\item Buscar o melhor modelo a ser utilizado especificamente para os dados de cardiologia no Brasil.
	\item Conduzir uma sequ?ncia de testes usando os dados dispon?veis para validar o modelo escolhido, se necess?rio, rever o modelo e refazer os testes.
	\item Discutir com especialistas na ?rea de cardiologia a validade do m?todo e seus resultados.
	
	%durante a conversa com o Miklos e a Dr. Li a pergunta especifica vai ficar bem mais clara, mas tem relacao entre o uso de drogas peri-operatorias e farmaco-genetica
	
\end{itemize}
Uma vez decidido o modelo a ser utilizado, esse deve permitir a entrada de informa??es fornecidas por especialistas e atualiza??o do modelo a partir do dados. Dessa forma, um modelo Bayesiano ? considerado, j? que as duas etapas podem ser utilizadas no modelo da seguinte forma:
 \begin{enumerate} 
	\item Entrada da informa??o \textit{a priori}: o modelos ? parametrizado de acordo com a informa??es dadas pelos especialistas.
	\item \textit{Calibra??o} do modelo: a partir da entradas dos dados a partir dos bancos de dados existentes, o modelo ? calibrado (os par?metros s?o ajustados para que o modelo se adeque aos dados).
 \end{enumerate}
A cria??o de tal modelo tem como objetivo final garantir que a melhor pr?tica na tomada de decis?o seja utilizada em hospitais brasileiros e que esse modelo contenha as caracter?stica locais do pa?s. Para isso, contaremos com a dissemina??o por parte de parceiros envolvidos no projeto, que est?o diretamente envolvidos com cardiologia no Brasil.\\
O modelo resultante e sua dissemina??o deve ser trazido para 
uma ampla discu??o envolvendo membros de hospitais de diversas ?rea do Brasil. Esse membros devem proceder ? criticar e propor mudan?as ao modelo at? que um modelo final que satisfa?a as necessidades de seus grupos seja encontrado e esse possa ser passado para os membros de seus respectivos grupos. O modelo deve apresentado na confer?ncia anual da Sociedade Brasileira de Cardiologia.\\

%por favor peca pra que a Clarissa te coloque em contato com o Dr. Bacal do INCOR para que ele possa ser incluido no projeto e se torne a conexao com a SBC

Apesar do projeto ser primeiramente focado apenas em problemas de cardiologia, se espera que tenhamos ao final do projeto uma metodologia que possa ser estendida a outras disciplinas e outros modelos possam ser criados para auxiliar m?dicos de outras ?rea em hospitais brasileiros.
\subsection{ \label{subsec:modelos} Modelos}
Os modelos a serem estudados e aplicados aos dados devem incluir:
\begin{enumerate}
\item ?rvores de decis?o 
\item Redes Bayesianas 
\item Campos aleat?rios de Markov
\item Modelo oculto de Markov
\end{enumerate}
Esse modelos t?m como vantagem, o fato de apresentarem as duas caracter?sticas desejadas: possibilidade de entradas de informa??es dos especialitas \textit{a priori} e possibilidade de atualiza??o do modelo a partir de \textit{aprendizado} utilizando dos dados dispon?veis e novos dados que podem ser entrados no sistema.\\
{\color{red} ... acrescentar info ...}\\
Usando esse modelo ? poss?vel fazer avalia??o da qualidade dos servi?os prestados, an?lise custo-benef?cio na gest?o de recursos p?blicos al?m de pesquisa de ader?ncia ?s diretrizes da Sociedade Brasileira de Cardiologia em hospitais da rede p?blica e privada.

\subsection{ \label{sec:resultados} Resultados Esperados}
Ao final do programa devem-se ter modelos de an?lises de decis?o relacionados a problemas cardiovasculares, para auxiliar m?dicos em hospitais brasileiros. Al?m disso deve ser criada uma metodologia de pesquisa, desde as etapas iniciais at? a aplica??o desses m?todos, para poss?vel reprodu??o da pesquisa e projeto em outras ?reas da medicina brasileira.

\subsection{ \label{sec:cronograma} Cronograma}
O cronograma do doutorado sadu?che inclue 12 meses na Universidade de Duke, divididos da seguinte forma:\\
{\color{red} ... acrescentar info ...}

%isso daqui nos vamos decidir durante a reuniao. uma vez que voce tenha um manuscrito mais completo dai eu reviso com voce de novo

\clearpage

\begin{thebibliography}{9}
\bibitem{bocchi}
Bocchi EA, Marcondes-Braga FG, Bacal F, Ferraz AS, Albuquerque D, Rodrigues D, et al. Sociedade Brasileira de Cardiologia. Atualiza??o da Diretriz Brasileira de Insufici?ncia Card?aca Cr?nica - 2012. Arq Bras Cardiol 2012: 98(1 supl. 1): 1-33
\bibitem{montera}
Montera MW, Almeida RA, Tinoco EM, Rocha RM, Moura LZ, R?a-Neto A, et al.
Sociedade Brasileira de Cardiologia. II Diretriz Brasileira de Insufici?ncia Card?aca Aguda.
Arq Bras Cardiol.2009;93(3 supl.3):1-65
\end{thebibliography}

\end{document}